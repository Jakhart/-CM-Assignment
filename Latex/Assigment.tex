\documentclass [10 pt, a4 paper]{article}
\usepackage{booktabs}
\usepackage[utf8]{inputenc}
\usepackage{graphicx}
\usepackage{array}
\author{Vincent Richard}
\date{02/11/17}
\title{Computtional Methods and C++ Assigment}

\begin{document}

\begin{titlepage}
    \maketitle
\end{titlepage}
\newpage

\begin{abstract}
    Sample text
\end{abstract}

\tableofcontents
\listoffigures
\listoftables
\newpage

%%%%%%%%%%%%%%%%%%%%%%%%%%%%%%%%%    Introduction    %%%%%%%%%%%%%%%%%%%%%%%%%%%%%%%%%%%%%%%%%

\section{Introduction}
In this assignment you are asked to examine the application of numerical schemes
for the solution of partial diferential equations. In order to  do so we will consider 
the following probem.

A wall 1 ft. thick and infinite in other directions has an initial uniform temperature Tin of 100$^{\circ}$F. The surface temperatures Tsur at the two
sides are suddenly increased and maintained at 300$^{\circ}$F. The wall is composed of
nickel steel (40\% Ni) with a diffusivity of D = 0:1 ft2=hr. Please compute the
temperature distribution within the wall as a function of time.
The governing equation to be solved is the unsteady one-space dimensional
heat conduction equation, which in Cartesian coordinates is:
\begin{equation}
    \frac{\partial T}{\partial t} = D \frac{\partial T}{\partial x^{2}}
\end{equation}

\subsection{Presentation of the different methods used}

\subsubsection{DuFort-Frankel}
The DuFort-Frankel scheme is an explicit scheme unconditionnaly stable the parabolic PDE is:
\begin{equation}
    \frac{T_{i}^{n+1} - T_{i}^{n-1}}{2\Delta t} = D \frac{T_{i+1}^{n} -(T_{i}^{n+1} + T_{i}^{n-1}) + T_{i-1}^{n})}{\Delta x^{2}}
\end{equation}
This equation leads to an explicit form which is:
\begin{equation}
    T_{i}^{n+1}(1 + 2r) = T_{i}^{n-1} +2r(T_{i+1}^{n} - T_{i}^{n-1} + T_{i-1}^{n}), r =\frac{D\Delta t}{\Delta x^{2}}
\end{equation}

\subsubsection{Richardson}
The Richardson scheme is an explixit scheme, unconditionnaly unstable:
\begin{equation}
    \frac{T_{i}^{n+1} - T_{i}^{n-1}}{2\Delta t} = D \frac{T_{i+1}^{n} - 2 T_{i}^{n} + T_{i-1}^{n}}{\Delta x^{2}}
\end{equation}
This equation leads to an explicit form which is:
\begin{equation}
    T_{i}^{n+1} = 2r(T_{i+1}^{n} - 2T_{i}^{n} + T_{i-1}^{n}) + T_{i}^{n-1}, r=\frac{D\Delta t}{\Delta x^{2}}
\end{equation}
%If we substitute the Fourier  mode $T_{i}^{n} = \lambda_{k}^{n}e^{ik\pi j\Delta x}$ into the scheme we are lead to the characteristic equation of the Richardson scheme:
%\begin{equation}
%    \lambda_{k}^{2}+8\lambda_{k}\mu sin^{2}(\frac{k\pi \Delta x}{2}) - 1 = 0
%\end{equation}

\subsubsection{Laasonen}
The Laasonen scheme is an implicit scheme, that as for equation:
\begin{equation}
    \frac{T_{i}^{n+1} - T_{i}^{n-1}}{2\Delta t} = D\frac{T_{i+1}^{n} - 2T_{i}^{n} + T_{i-1}^{n}}{\Delta x^{2}}
\end{equation}

This equation leads to a form that result in asystem of linear equation:
\begin{equation}
    -r T_{i+1}^{n+1} + (1+2r)T_{i}^{n+1} -rT_{i-1}^{n+1} =T_{i}^{n}, r=\frac{D\Delta t}{\Delta x^{2}}
\end{equation}

\subsubsection{Cranck-Nicholson}
The Crank-Nicholson scheme is an implicit scheme, that as for equation:
\begin{equation} 
    \frac{T_{i}^{n+1} - T_{i}^{n}}{\Delta t} = \frac{D}{2}(\frac{T_{i+1}^{n+1}-2T_{i}^{n+1}+T_{i-1}^{n+1}}{\Delta x^{2}} + \frac{T_{i+1}^{n}-2T_{i}^{n}+T_{i-1}^{n}}{\Delta x^{2}})
\end{equation}

This equation leads to a form that result in asystem of linear equation:
\begin{equation}
    -\frac{r}{2} T_{i+1}^{n+1}+(1+r)T_{i}^{n+1}-\frac{r}{2}T_{i-1}^{n+1} = \frac{r}{2}T_{i+1}^{n} + (1-r)T_{i}^{n} + \frac{r}{2}T_{i-1}^{n}
\end{equation}
\quad

%%%%%%%%%%%%%%%%%%%%%%%%%%%%%     Methods Procedures      %%%%%%%%%%%%%%%%%%%%%%%%%%%%%%%%%%%%%

\section{Methods and Procedures}
 


\end{document}

